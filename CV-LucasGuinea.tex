%%%%%%%%%%%%%%%%%
% This is an sample CV template created using altacv.cls
% (v1.7, 9 August 2023) written by LianTze Lim (liantze@gmail.com). Compiles with pdfLaTeX, XeLaTeX and LuaLaTeX.
%
%% It may be distributed and/or modified under the
%% conditions of the LaTeX Project Public License, either version 1.3
%% of this license or (at your option) any later version.
%% The latest version of this license is in
%%    http://www.latex-project.org/lppl.txt
%% and version 1.3 or later is part of all distributions of LaTeX
%% version 2003/12/01 or later.
%%%%%%%%%%%%%%%%

%% Use the "normalphoto" option if you want a normal photo instead of cropped to a circle
% \documentclass[10pt,a4paper,normalphoto]{altacv}

\documentclass[10pt,a4paper,ragged2e,withhyper]{altacv}
%% AltaCV uses the fontawesome5 and packages.
%% See http://texdoc.net/pkg/fontawesome5 for full list of symbols.

% Change the page layout if you need to
\geometry{left=1.25cm,right=1.25cm,top=1.5cm,bottom=1.5cm,columnsep=1.2cm}

% The paracol package lets you typeset columns of text in parallel
\usepackage{paracol}

% Change the font if you want to, depending on whether
% you're using pdflatex or xelatex/lualatex
% WHEN COMPILING WITH XELATEX PLEASE USE
% xelatex -shell-escape -output-driver="xdvipdfmx -z 0" sample.tex
\ifxetexorluatex
  % If using xelatex or lualatex:
  \setmainfont{Roboto Slab}
  \setsansfont{Lato}
  \renewcommand{\familydefault}{\sfdefault}
\else
  % If using pdflatex:
  \usepackage[rm]{roboto}
  \usepackage[defaultsans]{lato}
  % \usepackage{sourcesanspro}
  \renewcommand{\familydefault}{\sfdefault}
\fi

% Change the colours if you want to
\definecolor{SlateGrey}{HTML}{2E2E2E}
\definecolor{LightGrey}{HTML}{666666}
\definecolor{DarkPastelRed}{HTML}{450808}
\definecolor{PastelRed}{HTML}{8F0D0D}
\definecolor{GoldenEarth}{HTML}{E7D192}
\colorlet{name}{black}
\colorlet{tagline}{PastelRed}
\colorlet{heading}{DarkPastelRed}
\colorlet{headingrule}{GoldenEarth}
\colorlet{subheading}{PastelRed}
\colorlet{accent}{PastelRed}
\colorlet{emphasis}{SlateGrey}
\colorlet{body}{LightGrey}

% Change some fonts, if necessary
\renewcommand{\namefont}{\Huge\rmfamily\bfseries}
\renewcommand{\personalinfofont}{\footnotesize}
\renewcommand{\cvsectionfont}{\LARGE\rmfamily\bfseries}
\renewcommand{\cvsubsectionfont}{\large\bfseries}


% Change the bullets for itemize and rating marker
% for \cvskill if you want to
\renewcommand{\cvItemMarker}{{\small\textbullet}}
\renewcommand{\cvRatingMarker}{\faCircle}
% ...and the markers for the date/location for \cvevent
% \renewcommand{\cvDateMarker}{\faCalendar*[regular]}
% \renewcommand{\cvLocationMarker}{\faMapMarker*}


% If your CV/résumé is in a language other than English,
% then you probably want to change these so that when you
% copy-paste from the PDF or run pdftotext, the location
% and date marker icons for \cvevent will paste as correct
% translations. For example Spanish:
% \renewcommand{\locationname}{Ubicación}
% \renewcommand{\datename}{Fecha}


%% Use (and optionally edit if necessary) this .tex if you
%% want to use an author-year reference style like APA(6)
%% for your publication list
% \input{pubs-authoryear.tex}

%% Use (and optionally edit if necessary) this .tex if you
%% want an originally numerical reference style like IEEE
%% for your publication list
\input{pubs-num.tex}

%% sample.bib contains your publications
\addbibresource{sample.bib}

\begin{document}
\name{Lucas Guinea }
\tagline{Estudiante de Ingeniería Industrial}
%% You can add multiple photos on the left or right
\photoR{2.8cm}{LUCAS.jpeg}
% \photoL{2.5cm}{Yacht_High,Suitcase_High}

\personalinfo{%
  % Not all of these are required!
  \email{luacsguinearitta@gmail.com}
  \phone{+54-9-261-416-7551}
  \mailaddress{Carriego 643, Guaymallen}
  \location{Mendoza, ARGENTINA}
  \twitter{@LucasGuinea1}
  \linkedin{Lucas Guinea}
  \github{LucasGuinea}
  %% You can add your own arbitrary detail with
  %% \printinfo{symbol}{detail}[optional hyperlink prefix]
  % \printinfo{\faPaw}{Hey ho!}[https://example.com/]

  %% Or you can declare your own field with
  %% \NewInfoFiled{fieldname}{symbol}[optional hyperlink prefix] and use it:
  % \NewInfoField{gitlab}{\faGitlab}[https://gitlab.com/]
  % \gitlab{your_id}
  %%
  %% For services and platforms like Mastodon where there isn't a
  %% straightforward relation between the user ID/nickname and the hyperlink,
  %% you can use \printinfo directly e.g.
  % \printinfo{\faMastodon}{@username@instace}[https://instance.url/@username]
  %% But if you absolutely want to create new dedicated info fields for
  %% such platforms, then use \NewInfoField* with a star:
  % \NewInfoField*{mastodon}{\faMastodon}
  %% then you can use \mastodon, with TWO arguments where the 2nd argument is
  %% the full hyperlink.
  % \mastodon{@username@instance}{https://instance.url/@username}
}

\makecvheader
%% Depending on your tastes, you may want to make fonts of itemize environments slightly smaller
% \AtBeginEnvironment{itemize}{\small}

%% Set the left/right column width ratio to 6:4.
\columnratio{0.6}

% Start a 2-column paracol. Both the left and right columns will automatically
% break across pages if things get too long.
\begin{paracol}{2}
\cvsection{Experiencia}

\cvevent{Gerente asistente}{Transportadora de Cuyo}{enero 2019 -- Actualidad}{Mendoza, AR}
\begin{itemize}
\item Transporte,logística, cadena de suministros y almaceamiento.
\end{itemize}




\cvsection{Proyecto}

\cvevent{Auspicios \& Sales}{AArEII}{2021 -- 2022}{Mendoza, AR}
\begin{itemize}
\item Parte del equipo organizador de la JOSEII 2021
\item Parte del equipo organizador del CAEII MENDOZA 2022
\end{itemize}




\medskip

\cvsection{Un Día de mi Vida}

% Adapted from @Jake's answer from http://tex.stackexchange.com/a/82729/226
% \wheelchart{outer radius}{inner radius}{
% comma-separated list of value/text width/color/detail}
\wheelchart{1.5cm}{0.5cm}{%
  6/8em/accent!30/{Dormir},
  3/8em/accent!40/Curso y osio,
  8/8em/accent!60/Facultad,
  2/10em/accent/Gym ,
  5/6em/accent!20/Tiempo de calidad con mi familia
}
\cvsection{Education}

\cvevent{Ingeniería Industrial}{Universidad Nacional de Cuyo}{2019 -- Actualidad}{}

\divider

\cvevent{Bachillerato de economia}{I.S.E.P.}{2013 -- 2018}{}

\divider

% \divider
% use ONLY \newpage if you want to force a page break for
% ONLY the current column

%% Switch to the right column. This will now automatically move to the second
%% page if the content is too long.
\switchcolumn

\cvsection{Mi filosofia}

\begin{quote}
``Dar lo mejor y estar preparado para lo peor.''
\end{quote}

\cvsection{COMPETENCIAS CLAVE}

\cvachievement{\faMale}{Sociabilidad y Compañerismo}{Para ayudar a todo el que lo necesite}

\divider

\cvachievement{\faUsers}{Uso de herramientas informáticas}{Para sintonizar con las nuevas tendencias tecnológicas}

\divider

\cvachievement{\faChartLine}{Resolución de problemas}{Gran uso del sentido comun para entender problemas y encontrar la solución}

\divider

\cvachievement{\faCircle}{Organización}{Gran capacidad para segmentar mis días y poder afrontar todos mis objetivos}


\cvsection{FORTALEZAS}

\cvtag{Hard-working}
\cvtag{Detallista}\\
\cvtag{Responsable}\\
\cvtag{Motivador \& Lider}

\divider\smallskip

\cvtag{Pyton}
\cvtag{Manejo de Redes Socuakes}\\
\cvtag{Latex \&.dot}

\cvsection{Languages}

\cvskill{Ingles}{3.5}
\divider

\cvskill{Español}{5}
\divider

\cvskill{Portugues}{1.5} %% Supports X.5 values.

%% Yeah I didn't spend too much time making all the
%% spacing consistent... sorry. Use \smallskip, \medskip,
%% \bigskip, \vspace etc to make adjustments.
\medskip

\cvsection{Referees}

% \cvref{name}{email}{mailing address}
\cvref{Dueño\ Armando Buso}{TDC}{sbuso@live.com.ar}
{+54 9 261694-7184\\https://tdcuyo.com.ar}



% \divider

\end{paracol}


\end{document}
